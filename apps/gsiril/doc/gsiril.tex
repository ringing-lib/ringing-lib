\documentclass[a4paper,11pt,oneside]{book}

%\usepackage{lmodern}

% We're using xelatex: configure for unicode fonts
\usepackage{xltxtra}
\usepackage{xunicode}

% The cm-default option to fontspec is needed to render math symbols
% in certain versions of fontspec, and breaks with xltxtra in others.
%\usepackage[cm-default]{fontspec}

% Computer Modern warps my fragile little eyes
\usepackage{fontspec}
\defaultfontfeatures{Ligatures=TeX,Scale=MatchLowercase}
\setromanfont{Linux Libertine O}
\setsansfont{Linux Biolinum O}
%\usepackage{libertine}


\chardef\vapost="0027

% Sans-serif chapter headings
\usepackage{sectsty}
\allsectionsfont{\rm\sffamily} % \rm disables bold

% Page header showing chapter
\usepackage{fancyhdr}
\pagestyle{fancy}
\fancyhead{}\fancyfoot{} % Clear
\renewcommand{\chaptermark}[1]{\markboth{#1}{}}
\fancyhead[L]{\nouppercase{\leftmark}}
\fancyhead[R]{\thepage}
\addtolength{\headheight}{2pt} % Default is based on 10pt, not 11pt

% Better verbtaim text
\usepackage{fancyvrb}
\RecustomVerbatimEnvironment{Verbatim}{Verbatim}
  {xleftmargin=2em,fontsize=\small}
\VerbatimFootnotes

\usepackage{amsmath}  % for tfrac
\usepackage{upquote}

% URLs
\usepackage{url}
\DeclareUrlCommand\url{\def\UrlLeft{<}\def\UrlRight{>}\urlstyle{tt}}

% We want an index
\usepackage{makeidx}
\usepackage{index}
\makeindex

% Tables with lines spanning multiple rows
\usepackage{tabularx}

%\usepackage{footnote}
\usepackage[marginal]{footmisc}
% Allow lists, etc. in footnotes
\usepackage{bigfoot}

% Don't use US-style dates
\usepackage[level,nodayofweek]{datetime}

\usepackage{rotating}
\usepackage{multirow}
\usepackage[small]{caption}

\usepackage{enumitem}

% Coloured links, etc.
\usepackage[xetex,%
            citecolor=black,filecolor=black,linkcolor=black,urlcolor=black,%
            pdfauthor={Richard A. Smith},%
            pdftitle={A user's guide to GSiril},%
            bookmarks,colorlinks,hyperfootnotes,hyperindex,unicode]{hyperref}

% Code to make indexes how we want them...
\newdimen\optwidth\newdimen\loptwidth\newdimen\fspecwidth
\setbox0=\hbox{\footnotesize\verb+-+}\optwidth=\wd0
\setbox0=\hbox{\footnotesize\verb+--+}\loptwidth=\wd0
\setbox0=\hbox{\footnotesize\verb+$+}\fspecwidth=\wd0
% \textitidx -- a definition formated in italics and indexed
\def\textitidx#1{\textit{#1}\index{#1}}
% \oid  -- option index defintion -- the main definition of a short option 
% \oi   -- option index -- a secondary reference to a short option
% \loid -- long option index definition -- the main definition for a long opt
% \loi  -- long option index -- a secondary reference to a long option
\newcommand{\ulink}[1]{\underline{\hyperpage{#1}}}
\newcommand{\oidx}[2]{\index{#1@{\hspace*{-\optwidth}\texttt{-}#2}|ulink}}
\newcommand{\oid}[2]{\oidx{#1}{\texttt{#1}}%
  \index{#2@{\hspace*{-\loptwidth}\texttt{--#2}}|see{\texttt{-#1}}}}
\newcommand{\oi}[1]{\index{#1@{\hspace*{-\optwidth}\texttt{-}\texttt{#1}}}}
\newcommand{\loid}[1]{\index{#1@{\hspace*{-\loptwidth}\texttt{--#1}}|ulink}}
\newcommand{\loi}[1]{\index{#1@{\hspace*{-\loptwidth}\texttt{--#1}}}}
\newcommand{\optdesc}[2]{\index{#2|see{\texttt{-#1}}}}
% \fspec[d] -- format specifier [definition]
\newcommand{\fspecd}[1]{\index{#1@{\hspace*{-\fspecwidth}\texttt{\$#1}}|ulink}}
\newcommand{\fspec}[1]{\index{#1@{\hspace*{-\fspecwidth}\texttt{\$#1}}}}
% symbol index
\newcommand{\symidx}[2]{} %{\index{{#1}@{\texttt{#1}}|see{#2}}}
\newcommand{\ttcmdidx}[1]{\texttt{#1}\index{#1@{\texttt{#1}}}}

% \gsiril -- the program name, in the preferred capitalisation
\def\gsiril{GSiril}
% \sref -- section ref
\newcommand{\sref}[1]{\hyperref[#1]{\S\ref{#1}}}

\def\half{\tfrac{1}{2}}
\def\quarter{\tfrac{1}{4}}

\makeatletter
\def\printidx#1{\@print@index[default][#1]}
\makeatother

% And now for something different...
\title{A user's guide to \gsiril}
\author{Richard A.\ Smith\\\url{richard@ex-parrot.com}}

\begin{document}

\frontmatter
\maketitle

\clearpage
\phantomsection
\addcontentsline{toc}{chapter}{Contents}
\tableofcontents
% General paragraph style -- this has to be after the table of contents
\parskip=1em
\parindent=0em
\addtolength{\footnotesep}{4pt}

\mainmatter

\chapter{Getting started}

This chapter gives a brief introduction to \gsiril\ -- what it is,
what it can do, and who it's aimed at (\sref{what_is_it}).  
This is followed by an introduction to the command line interface (\sref{cli}),
aimed particularly at those unfamiliar with it; it can safely be
skipped by those who are familiar with Unix-style command line interfaces 
or who intend to use \gsiril\ via the web interface on BellBoard.
Finally, there is a simple worked example (\sref{example}),
details on how to run \gsiril\ interactively (\sref{interactive}),
how to obtain \gsiril\ (\sref{obtain}), and its
current development status (\sref{dev_status}).


\section{What is it?}\label{what_is_it}

\gsiril\ is a program for proving bell ringing compositions. 
If you're not a bell ringer or don't know what a composition is, 
\gsiril\ almost certainly isn't for you.
Even if you are a ringer and do know what a composition is, this tool
is likely only to of interest if you have an interest in composition.

\gsiril\ is intended to be backwards compatible with
MicroSiril,\index{MicroSiril} a proof engine written by Sam
Wilmott\index{Wilmott, Sam} in 1987 which rapidly became the
\textit{de facto} standard; it, in turn, was based on ideas and
algorithms by Andrew Craddock\index{Craddock, Andrew} and Hamish
McNaughton\index{McNaughton, Hamish}. 

There are several reasons why I decided to write \gsiril\ rather than
using an existing program.  The first reason was very straightforward:
when I started work on \gsiril, all proving programs were designed to
run on Windows or DOS, but I wanted to be able to prove compositions on
Linux.  A second reason was that none of the other programs I tried did
exactly what I want.  Some programs couldn't cope with multi-extent
touches; other's were restricted to 12 or 16 bells.  Many used a syntax
that I thought was unnatural.  Finally, there was an ideological reason.
I am a great supporter of open source software, and back in 2002 there
was very little open source ringing software, and no composition prover.

\section{Command line interface}\label{cli}\index{command line interface}

\gsiril\ is a command line utility.  This means it does not have a
whizzy graphical user interface, and is instead run from your system's
\textit{command prompt}\index{command prompt|see{prompt}}%
\index{prompt} or \textitidx{terminal}.
\gsiril\ attempts to read from standard input,\index{standard input}
meaning it is possible to type the \gsiril\ language directly into the
program, but it is often more convenient to get \gsiril\ to read from a file.
This can be done via redirection,\index{shell!redirection}%
\index{redirection|see{shell, redirection}} which means writing the name of the
file after a \verb+<+ character, or using the `\verb+-f+' command line
option\oi{f}.  Both of the following lines will read the `\verb+filename.sir+'
file and try to prove the composition in it.

\begin{Verbatim}
gsiril < filename.sir
gsiril -f filename.sir
\end{Verbatim}

For users who are not comfortable using the command line interface,
\gsiril\ can also be used online on \textit{The Ringing World}'s
BellBoard\index{BellBoard} website:

\begin{quotation}
\url{https://bb.ringingworld.co.uk/gsiril}
\end{quotation}

\subsection{Using \gsiril\ on Windows}\index{Windows}

Most Windows users only rarely encounter programs with a command line
interface, but Windows does provide a terminal that allows you to use
\gsiril.  To start a terminal, you may find `Command Prompt' as an
option on your `Start' menu; if not, hold down the Windows key and press
`R' (for `Run'), then type in  `\ttcmdidx{cmd.exe}'.%
\footnote{On Windows 95, 98 or ME, this will be `\texttt{command.com}'.%
\index{command.com@\texttt{command.com}}}

You should then see a new window, probably black with a white font, 
containing something like the following.  (The exact details will 
depend on your version of Windows.)

\begin{Verbatim}
Microsoft Windows [Version 6.0.6000]
Copyright (c) 2006 Microsoft Corporation.  All rights reserved.

C:\Users\Richard>
\end{Verbatim}

The last line is called the prompt and indicates that the terminal is
waiting for you to type something in; the prompt should always end in a
`\verb+>+' character. %\symidx{>}{Windows prompt}
First you need to change into the directory that contains your copy of
\texttt{gsiril.exe} (plus any associated tools and libraries that may have
come with it).  For example, if you've put it in a directory called
`\verb+C:\Users\Richard\gsiril+', you should type:

\begin{Verbatim}
cd C:\Users\Richard\gsiril
\end{Verbatim}

In some versions of Windows, you can avoid this by
starting the command prompt in the right
directory by right clicking on the folder and choosing 
`Open Command Prompt Here' on the context menu.
You can now check that \gsiril\ is accessible by typing:

\begin{Verbatim}
gsiril --version
\end{Verbatim}
\oi{V}

After pressing enter, this should print some brief information about the 
version of \gsiril\ that you are running.

\subsection{Help!}\label{help}

As with most command line utilities, you can obtain a brief summary 
of valid options to \gsiril\ by using the \verb+--help+ option.%

\begin{Verbatim}
gsiril --help
\end{Verbatim}
\loi{help}

This user's guide aims to provide more extensive documentation as 
well as some worked examples of how it can be used in real world situations.
Don't be afraid to experiment.  If you're still stuck, drop me an email:
my email address is on the title page.

\section{An example — proving Middleton's Cambridge}\label{example}
\index{example|(}

Suppose I want to prove Charles Middleton's\index{Middleton, Charles} 
well-known 5600 of Cambridge Surprise Major.  First, I create a text file 
called \texttt{middleton.sir} containing the following description of
the peal. 

\begin{Verbatim}
8 bells

Cambridge = &-3-4-25-36-4-5-6-7

p = Cambridge, +2
b = Cambridge, +4

M  = p,p,b,p,p,p,p
MW = p,p,b,b,p,p,p
WH = p,p,p,b,p,p,b
H  = p,p,p,p,p,p,b

part = M, MW, WH, H, H

prove 5 part
\end{Verbatim}

To run this using the web interface, simply paste this code into the
large composition box and hit the `Prove' button; or to run it from the
command prompt, just type\oi{f}:

\begin{Verbatim}
gsiril -f middleton.sir
\end{Verbatim}

\gsiril\ will then prove the touch and print the following message to
confirm that it is true.

\begin{Verbatim}
5600 rows ending in 12345678
Touch is true
\end{Verbatim}

Had there been an error in the composition which resulted in the composition
being false, \gsiril\ would have said something like:

\begin{Verbatim}
288 rows ending in 17864523
Touch not completed due to false row
\end{Verbatim}


\subsection{A closer look}

The first line of this file is `\texttt{8 bells}', which tells 
\gsiril\ that any place notation is to be interpreted as on eight bells,
and the composition is to be proved on eight bells.

The next eight non-blank lines define a series of symbols which
represent the blocks of changes from which the peal is built.  The first
symbol, `\texttt{Cambridge}' is defined using a piece of place notation
for one lead of Cambridge Surprise Major, up to but excluding the lead
end change.  The `\verb+&+' at the beginning of the place notation tells
\gsiril\ that the block is palindromic and that only the first half has
been given explicitly, so even though only 16 changes have been given,
this represents a block of 31 changes.\footnote{Written out in full, this
block represents the place notation \texttt{-38-14-1258-36-14-58-16-78
-16-58-14-36-1258-14-38-}.}

The next two symbols are `\texttt{p}' and `\texttt{b}', which have been
defined as a plain lead and a bobbed lead of the method, respectively, by
appending a seconds or fourths place lead end change to the method.  The
lead end changes are written is place notation, starting with a
`\verb!+!' which tells \gsiril\ that the place notation has been given
in full.  All blocks of place notation in \gsiril\ must be preceded with
either `\verb+&+' or `\verb!+!' so \gsiril\ recognises them as blocks of
place notation.

Next, the symbols `\texttt{M}', `\texttt{MW}', `\texttt{WH}' and
`\texttt{H}' represent the four different types of course used in the
peal, with various combinations of bobs at the middle, wrong
and home calling positions, which in Cambridge come at the third, fourth
and seventh lead ends, respectively.  Each is defined as a sequence of
seven plain or bobbed leads.  Finally, `\texttt{part}' is defined as a
sequence of five courses.

The final line of the file is `\texttt{prove 5 part}'.  The
`\texttt{prove}' keyword tells \gsiril\ that is a prove statement.  In
this case, it proves the block represented by `\texttt{part}' rung five
times consecutively and prints some basic output saying how long it is,
whether it is true (meaning no row occurs more than once) and whether it
comes round.

\index{example|)}

\section{Interactive mode}
\label{interactive}

Unless \gsiril\ is told not to,\footnote{The \verb+-f+,\oi{f}
\verb+-e+\oi{e} and \verb+-N+\oi{N} tell \gsiril\ not to read source
from standard input.  See \sref{inputopts}.} it will read input from standard
input.\index{standard input} This means it is possible to type the \gsiril\ 
language directly into the program.  When doing this, it can be useful to the
\verb+-i+ and \verb+-v+ options.

The \verb+-i+ option\oid{i}{interactive} puts \gsiril\ into interactive
mode. This does two things: errors in user input do not cause
\gsiril\ to exit immediately, and on systems that support it, it enables
more user-friendly line editing,\footnote{Specifically, if \gsiril\ was
compiled with support for GNU readline\index{readline}, the cursor keys
will scroll backwards and forwards through previous lines, allowing previous
lines to be edited and re-run, for example to correct mistakes.} and displays
a \textitidx{prompt} of `\verb+>+' whenever it is waiting for input.
The \verb+-v+ option\oid{v}{verbose} makes \gsiril\ more verbose, and it
will provide confirmation of what it has done after each command.

\gsiril\ will continue accepting input until it reaches the end of the
input stream.  When typing directly into \gsiril, you can tell \gsiril\ that
you have finished by pressing Control-D on Linux, or Control-Z on Windows.
Alternatively, a statement containing just the single word \ttcmdidx{end},
\ttcmdidx{quit} or \ttcmdidx{exit}, will cause \gsiril\ to stop taking
input and exit. 

The following example shows a \gsiril\ session in interactive mode,
instigated by running \texttt{gsiril -i -v}, which is used to prove a
plain course of Double Court Bob Minor.  The prompt (the leading
`\verb+>+') makes it clear which lines have been entered by the user,
and which are \gsiril's responses.

\begin{Verbatim}
Ringing Class Library / gsiril 0.4.0-pre
GSiril is licensed under the GNU General Public License, v2+
> 6 bells
Set bells to 6
> m = &-4-3-6,+6
Definition of 'm' added.
> prove 5m
60 rows ending in 123456
Touch is true
> exit
\end{Verbatim}


\section{Obtaining \gsiril}\label{obtain}

The recommended way of obtaining \gsiril\ is to download the source code 
from GitHub\index{GitHub} and compile it yourself.  Such things are relatively
easy under Linux where most distributions will provide the necessary tools as
standard.  However Windows users will likely find this process harder.  To
compound matters, whilst the source code should be portable to a whole 
range of different compilers and operating systems, life is rarely that 
straightforward, and in all likelihood if you're not using a combination
that I've tested, you'll end up having to fix a handful of small problems.

\index{compiling gsiril@compiling \gsiril}

To compile from source, checkout the code from GitHub\index{git},
generate the \verb+configure+ script using the \verb+autoreconf+
command, and then configure, compile and install it in the
standard way:
\begin{Verbatim}
git clone https://github.com/ringing-lib/ringing-lib
cd ringing-lib
autoreconf
./configure --prefix=$HOME
make
make install
\end{Verbatim}

This should install the Ringing Class Library\index{Ringing Class Library} 
and all of its associated programs (including \gsiril) into your home 
directory.  Depending how your system is configured, you may need to add this
to your \textit{path} so that the shell can find
\gsiril.  If, when trying to run \gsiril, you see an error message
saying `\texttt{command not found}', you 
probably need to run the following:\footnote{Or add it to your 
\verb+~/.bashrc+ so they are always available.}\fspec{PATH}
\begin{Verbatim}
export PATH=$PATH:$HOME/bin
\end{Verbatim}

\section{Development status}
\label{dev_status}

\gsiril\ is a fairly mature piece of software and has been in active
use since 2002, since which time a great many new features have been added.
Some features will have been tested extensively in many different 
combinations and are likely to be bug-free; however, other features are
still very new and there may well be bugs them, and some of the less
commonly used features may interact with each other in odd ways.
If you find something which you believe to be a bug,%
\index{bugs, reporting} do please contact me, either by email or by filing an
issue on GitHub.

\gsiril\ is also a work in progress, and there is still a lot of
functionality that I would like to add.  Generally speaking, new
functionality gets added as and when I want it or someone requests it,
assuming I can see a suitable way of implementing it.  If there is
missing functionality that you would particularly like to see added,
drop me an email and I'll see what I can do about adding it.

Finally, \gsiril\ is open source software.\index{open source}
This means that anyone 
can download a copy of the source code and extend it.%
\footnote{The source code for \gsiril\ is included with the 
Ringing Class Library which can be downloaded from 
\url{https://github.com/ringing-lib/ringing-lib}.}
For example, I have no interest in writing a graphical user interface to
\gsiril, besides the web interface on BellBoard, but if you're a
programmer and want to, there is nothing to prevent you from doing so.
All that is asked in return is that if you make any extensions publicly
available, you also make the source code to them freely available.%
\footnote{Specifically, \gsiril\ is covered by the GNU General Public 
License, version 2 or above,\index{licence} which can be viewed at
\url{https://www.gnu.org/licenses/old-licenses/gpl-2.0.html}.}

\chapter{Statements}

This chapter describes the general structure of a \gsiril\ file 
(\sref{stmt_struct}), and then documents each type of statement allowed
in the file.  
The formal grammar for statements can be found in Appendix
\ref{stmt_grammar}.

\section{The basic structure}
\label{stmt_struct}

A \gsiril\ file consists of a sequence of \textitidx{statements}, each
of which is an instruction to \gsiril\ to do something.

\index{line breaks|(}
Several statements may be placed on one line, in which case they must be
separated by semicolons.  A line break may also used to mark the end of
a statement, and a semicolon at the end of the line is optional.  For
example, the following line contains two statements separated by a
semicolon; the optional semicolon at the end of the line has been
omitted here.

\begin{Verbatim}
p = m,+12; b = m,+14
\end{Verbatim}

Because a line break is used to terminate a statement, it is not
possible to break long statements into multiple lines arbitrarily.
However, to allow statements to be more conveniently formatted, a line
break is ignored if it is immediately after a comma,\footnote{This is
possible because a comma can never be the last character in a statement,
as it is a binary operator whose second argument is mandatory.} or
anywhere within parentheses or braces.  For instance, the
following code contains a single statement broken across two lines.

\begin{Verbatim}
prove 3( Cm,b,Cm,Pr,Pr,Cm,
         Cm,b,Cm,Pr,Pr,Cm,b );
\end{Verbatim}

The line break here occurs after a comma and also within parentheses,
either of which alone would have been sufficient for \gsiril\ to ignore
it.\index{line breaks|)}

Comments\index{comments} provide a way of including a human-readable
explanation or annotation in the machine-readable \gsiril\ code.  
Comments start with a pair of slashes and continue to the end of the
line.\footnote{In MicroSiril compatibility mode, a single slash is
sufficient.  See \sref{compatibility}.}  They are allowed anywhere a
line break is allowed, and are ignored by \gsiril.

\begin{Verbatim}
m = &-3-4-25-36-4-5-6-7  // Cambridge
\end{Verbatim}

\section{Definitions}\label{defnstmt}
\index{definitions|(}

Definitions are a way of assigning a symbolic name to an expression,
such as a block of place notation, so that it can be used in subsequent
statements.  The normal syntax for a definition statement is:

\begin{quote}
\textit{name} \;\verb+=+\; \textit{expression}
\end{quote}

The name must start with a letter, and may contain letters, numbers or
underscores.\footnote{In MicroSiril compatibility mode, the name may
also contain (but not begin with) a `\verb+!+', `\verb+-+' or
`\verb+%+'.  See \sref{compatibility}.} Unless the \verb+-I+\oi{I}
command line option has been used, symbol names are case sensitive. 

The right-hand side is an expression and is documented in
\sref{expr}.  The right-hand side may be omitted, in which case the
statement defines the symbol to be the null expression.%
\footnote{This is not the same as undefining the symbol.  A symbol
defined to be the null expression may be used and does nothing, but
attempting to use an undefined symbol results in an error.}
The following example shows some simple definitions for plain and bobbed
leads of Bristol Major.  Other than the final change, these are
identical, so the common sequence of changes is factored out into a
symbol called \texttt{m}.

\begin{Verbatim}
m = &-5-4.5-5.36.4-4.5-4-1;
p = m, +8;
b = m, +4;
\end{Verbatim}

It is not an error to define a symbol more than once: subsequent
definitions replace earlier ones in the symbol table.  If the assignment
operator (`\verb+=+') in a definition statement is replaced with a
conditional assignment\index{conditional assignment} operator
(`\verb+?=+'), this only defines the symbol if it was not already
defined.
\index{definitions|)}

\section{Configuration statements}
\label{configstmts}

Before \gsiril\ proves a touch, it needs to know three things: the
number of bells, the number of times each row is allowed, and which row
to start from.  These pieces of information can be supplied using
command line options per \sref{cmdopts}, or they can be set using
\texttt{bells}, \texttt{extents} and \texttt{rounds} statements.  The
\texttt{rows} and \texttt{row\_mask} statements have related roles
and are also documented here.

The \ttcmdidx{bells} statement takes the form of a number followed by
the word `\texttt{bells}'.  It tells \gsiril\ to interpret place
notation and row literals as having that number of bells, the same
function as the \verb+-b+\oi{b} option.  There is no default number of
bells, and either a \texttt{bells} statement or a \verb+-b+ option is
required before place notation or row literals can be used.  The number
of bells may be set multiple times in a source file.\footnote{The can be
useful when proving variable cover touches, so that the method can be
defined using its normal place notation without having to add the places
made by the covering bell, and then the number of bells changed to
define the call which change the covering bell.}

The \ttcmdidx{extents} statement also takes the form of a number followed by
the word `\texttt{extents}'.  It tells \gsiril\ how many times to allow
a row to appear before considering the touch to be false, which is the
same purpose as the \verb+-n+\oi{n} option.  By default rows are only
allowed to occur once, so an \texttt{extents} statement or a \verb+-n+
option is only needed when proving multi-extent blocks.

The \ttcmdidx{rounds} statement takes the form of the word
`\texttt{rounds}' followed by a row literal, which is simply a row
written in single quotes.  It serves the same purpose as the \verb+-r+
option, and tells \gsiril\ to use a different row instead of the rounds
as the starting row.

These three statements are illustrated in the following example:

\begin{Verbatim}
6 bells;          // Sets the number of bells to 6
2 extents;        // The touch is allowed to contain rows twice
rounds '134256';  // Start the touch from 134256
\end{Verbatim}

The \ttcmdidx{rows} statement can be used to specify the expected
length of the touch, either as an exact value or as a range.  The
\verb+--length+\loi{length} option serves this purpose too.  If touch is
too long or too short, an error is generated.  This statement takes the
form of either a single number or a pair of numbers separated by the
word `\texttt{to}', followed by the word `\texttt{rows}'.  The former
represents an exact length, while the latter specifies a length
range using the minimum and maximum permitted values.  For example:

\begin{Verbatim}
720 rows;
1250 to 1450 rows;
\end{Verbatim}

The \ttcmdidx{row\_mask} statement is used to alter how rows are printed
by \gsiril.  It takes the form of the word `\texttt{row\_mask}' followed
by a row pattern, which is written between slashes.  Whenever a row is
printed using a string like `\texttt{"@"}', the row is matched against
the row mask pattern; if it matches, only those bells which match a
wildcard (i.e.  \verb+?+ or \verb+*+) are printed;\footnote{A range like
`\verb+[56]+' does not count as a wildcard.} if it doesn't match,
the string is not printed.  This statement is not supported in
MicroSiril compatibility mode.  It is useful for suppressing fixed
bells\index{fixed bells} when only lead heads or course heads are
printed.  For example, the following file prints the lead heads of Plain
Bob Major, with the treble omitted:

\begin{Verbatim}
8 bells;
row_mask /1*/;
p = &-8-8-8-8,+2,"@";
prove 7p;
\end{Verbatim}

The \verb+--row-mask+ option\loi{row-mask} serves the same purpose.
The \texttt{row\_mask} statement has no effect if the \verb+-E+ option is
used, though the \verb+--row-mask+ option continues to work.

\section{Prove statements}

There are two ways to prove touches: using a \ttcmdidx{prove} statement,
or using the \verb+-P+ option which is described in \sref{compatibility}.

When entering input directly into \gsiril, or when there is more than one
touch to prove, the \texttt{prove} statement is often more useful.  This
statement simply consists of the word `\texttt{prove}' followed by the
expression to prove, which may be just a symbol name, or maybe a more
complex expression..  This example will prove the standard 720 of Plain
Bob Minor, assuming \texttt{p} and \texttt{b} have already been defined
appropriately.

\begin{Verbatim}
prove 3(b,3p,2b,4p);
\end{Verbatim}

\section{Miscellaneous statements}

The \ttcmdidx{import} statement provides \gsiril\ with a way of
including the contents of one file within another.  This can be useful
for building up files containing libraries of definitions.  An
\texttt{import} statement at the start of a file is equivalent to an
\verb+-m+\oi{m} option.  The syntax of the import statement is as
follows.

\begin{Verbatim}
import "filename"
\end{Verbatim}

While importing a file, whether using an \texttt{import} statement or an
\verb+m+ option, the \verb+-i+\oi{i} and \verb+-v+\oi{v} options are
ignored.  Amongst other things, this means an error in an imported file
always terminates processing of that file.  An imported file may set the
number of bells via a \texttt{bells} statement, but this setting only
persists if the number of bells had not already been set.

% TODO echo

The \ttcmdidx{version} statement simply prints the version of \gsiril.
Syntactically, it comprises just the word `\texttt{version}'.

\chapter{Expressions}
\label{expr}

\section{Operators}

\begin{tabularx}{\textwidth}{ll}
\textit{fn}\verb+(+\textit{x}\verb+,+\textit{y}\verb+,+\ldots\verb+)+&
  Function call\\
\verb+echo+, \verb+defined+, \verb+repeat+, \textit{n}, \verb=++=, \verb=--=,
\verb+~+, \verb+!+& Unary prefix operators\\
\verb+*+, \verb+.+, \verb+/+, \texttt{\%}& Multiplicative operators
(\sref{arithop})\\
\verb=+=, \verb=-=& Additive operators (\sref{arithop})\\
\verb+<<+, \verb+>>+& Shift operators\\
\verb+&+& Merge replacement operator\\
\verb+=+, \verb+!=+, \verb+<+, \verb+>+, \verb+<=+, \verb+>=+&
  Comparison operators (\sref{cmpop})\\
\verb+&&+& Logical `and' operator (\sref{logop})\\
\verb+||+& Logical `or' operator (\sref{logop})\\
\verb+,+& Comma operator\\
\verb+=+, \verb+.=+, \verb+:=+& Assignment operators
\end{tabularx}

\subsection{Comparison operators}\label{cmpop}
\index{comparison operator|(} 

The six comparison operators,
\verb+<+, \verb+>+, \verb+<=+, \verb+>=+, \verb+==+ and \verb+!=+ 
can be used to compare integers or strings.  The test whether the left-hand
side is less than, greater than, less than or equal to, greater than or equal
to, equal to, or not equal to the right-hand side, respectively.  Integers are
compared arithmetically\index{arithmetic comparison}, while strings are
compared lexicographically\index{lexicographic comparison}.  This means that
\verb+11>2+ is true, while \verb+"11" > "2"+ is false.

In addition, the \verb+==+ and \verb+!=+ operators can compare booleans,%
\footnote{On booleans, \verb+==+ and \verb+!=+ behave as exclusive nor and
exclusive or operators, respectively.%
\index{xor|see{exclusive or}}\index{exclusive or}} 
or an integer with a string.  When an integer is compared with a string, the
integer is first converted to a string with no leading zeros, and this is
compared lexicographically with the string argument -- thus \verb+01=="1"+
evaluates to true, but \verb+1=="01"+ evaluates to false.

The result of all six operators is a boolean.
\index{comparison operator|)} 

\subsection{Logical operators}\label{logop}
\index{logical operators|(}

The \verb+&&+ and \verb+||+ operators are the logical `and' and logical `or'
operators.\index{and operator}\index{or operator}  They are binary operators
whose argument is evaluated as a boolean.\footnote{Note that, unlike in many
languages, there is no conversion from an integer from a boolean.  If \verb+n+
is an integer, you cannot write \verb+n && x+.  You must instead write
something like \verb+n!=0 && x+.}
The \verb+&&+ operators evaluates to true if both of its arguments evaluate to
true; the \verb+||+ operator evaluates to true if either (or both) of its
arguments evaluate to true.  

Both of these operators evaluate their left-hand argument first and 
\textit{short circuit}\index{operators!short circuiting} 
based on the value of it --- this means that if
the result of the operator does not depend on value of the second argument,
then the second argument is not evaluated.  In practice this means that if
the first argument to \verb+||+ is true or if the 
the first argument to \verb+&&+ is false, then the second is not evaluated.
This is useful if evaluating the seconds argument would produce an error or
if it would be computationally expensive.

The \verb+!+ operator is the logical `not' operator.\index{not operator}  It
evalutes its argument and inverts it -- if the argument was true, it evaluates
to false, and \textit{vice versa}.

The result of all three operators is a boolean.

\index{logaical operators|)}

\subsection{Arithmetic operators}\label{arithop}
\index{arithmetic operator|(}

The \verb-+-, \verb+-+, \verb+*+ and \verb+/+ operators fulfil their
usual mathematical roles of addition\index{addition}, 
subtraction\index{subtraction}, multiplication\index{multiplication}
and division\index{division}, respectively, with the caveat that they
perform \textitidx{integer arithmetic}.  This is of particular relevance
to division where the result is rounded to the next integer towards zero ---
so, \verb+5/3+ evaluates to 1.

The \verb+%+ operator is the modulus operator\index{modulus operator}.
It returns the remainder left from the integer division of its arguments.
For example, \verb+8%3+ evaluates to 2.  It can be defined more mathematically
in terms of integer division by the equation:
\[ a \mathop{\verb+/+} b \mathop{\verb+*+} b 
   \;\mathop{\verb-+-}\; a \mathop{\verb+%+} b = a \]
Because integer division rounds towards zero rather than down, this means that
when $a$ is negative, the result of the modulus operator is also negative.%
\footnote{To those familiar with modular arithmetic, this may be surprising.
But this is behaviour is typical in most programming languages that use
integer arithmetic --- for example, the C language, and languages derived
from it, define the \verb+%+ operator in this way.}

All five operators are binary operators.  They evaluate both of their
arguments as integers and the result is an integer.

The \verb+*+ operator has a second meaning, which to do repetition.  This
behaviour is invoked if only one of its argument is an integer.
\index{arithmetic operator|)}


\chapter{Command line options}
\label{cmdopts}

The behaviour of \gsiril\ can be customised using \textit{command line
options}\index{command line option|see{options}}%
\symidx{-}{options}\index{options|(}, which are placed after the program
name on the command line.  A full list of available options can be got
by running \gsiril\ with the \verb+--help+ option:\loid{help} i.e.\ by
running `\texttt{gsiril --help}'.  This chapter documents these options.

\section{Using options}

Some options are \textit{boolean options}: either they are present or they are
not, and they are not parametrised in any way.  The \verb+-i+ and \verb+-v+
options which were introduced in \sref{interactive} are examples of boolean
options.  By contrast, the \verb+-b+ option takes a \textit{mandatory
argument}, namely the number of bells, while the \verb+-P+ option takes an
\textit{optional argument} to say which symbol to prove, and defaults to the
first symbol defined in the source.

By convention, command line options can sometimes be merged together.
This can be confusing to the uninitiated as it results in a much more terse
command invocations.  The rule is that any number of boolean options can be
concatenated, followed, optionally, by at most one option which takes an
argument.  For example, the following three command lines all behave the same:

\begin{Verbatim}
gsiril -i -v -b6
gsiril -iv -b6
gsiril -ivb6
\end{Verbatim}

Each command line option also has a \textit{long form}.  These all begin
with two hyphens rather than one, and have long names that are intended to 
make their meaning more obvious to a casual reader instead of the 
normal single-character ones.  The other difference is that when they
have arguments, the argument is preceded by an `\verb+=+'.  For example,
`\verb+-b6+' can be written `\verb+--bells=6+'.  
Certain rarely-used options only have a long form: for example,
the option to disable the \texttt{import} directive is 
\verb+--disable-import+ --- there is no equivalent short form.
Perhaps unsurprisingly, it is not possible to merge together the long forms 
of arguments in a way analogous to \verb+-ivb6+.

When an option's argument contains spaces or punctuation marks, it is often
necessary to enclose them in quotes.  This is not a requirement imposed by
\gsiril, but is often necessary to prevent these characters from being
interpreted specially by the shell.\index{quoting|see{shell, quotation}}
\index{shell!quotation}
On most Unix-like systems (including most Linux shells) single quotes are
usually better, unless the argument itself contains a single quote character,
in which case double quotes are necessary.\footnote{The disadvantage of double
quotes is that in most Unix shells, the characters \verb+$+, \verb+`+,
\verb+\+ and sometimes \verb+!+ will still be treated specially by the shell,
even though they have been quoted.  This behaviour can be suppressed by
placing a `\verb+\+' character before them, a process called escaping;
\index{escaping|see{shell, escaping}}\index{shell!escaping} 
a backslash can also be used to escape a double quote character, should one be
needed in the argument.
Using single quote characters removes the need to escape any characters to
prevent the shell from interpreting them specially.} 
Under the Windows command prompt, double quotes appear to be necessary.


\section{Controlling input}
\label{inputopts}

\begin{tabularx}{\textwidth}{llX}
\texttt{-e}&\texttt{--expression=STMTS}&
  Execute the statements \texttt{STMTS}\\
\texttt{-f}&\texttt{--script-file=FILENAME}&
  Execute file \texttt{FILENAME}\\
\texttt{-N}&\texttt{--no-read}&
  Don't read \gsiril\ source from standard input\\
\texttt{-D}&\texttt{--define=NAME=VALUE}&
  Define a particular symbol\\
\texttt{-m}&\texttt{--module=MODULE}&
  Import the given module\\
&\texttt{--no-init-file}&
  Do not real the standard initialisation file
\end{tabularx}

By default \gsiril\ reads source from standard input\index{standard input}.
The \verb+-e+, \verb+-f+ and \verb+-N+ options suppress that by making
\gsiril\ read source from elsewhere or nowhere.  At most one of these options
can be used on a command line.

The \verb+-e+\oid{e}{expression} option instructs \gsiril\ to use the
option's argument as the source to be proved.  Multiple statements can be
included separated by semicolons, exactly as in the input file.  Frequently
this option's argument will contain spaces or punctuation that will need
quoting.

The \verb+-f+\oid{f}{script-file} option provides an alternative to using 
redirection\index{shell!redirection} to tell \gsiril\ to read source from the
named file.  

The \verb+-N+\oid{N}{no-read} option tells \gsiril\ not to read input
from standard input in the absence of an \verb+-f+ or \verb+-e+ option.
This can be useful when testing modules.  This option does not prevent
\gsiril\ reading methods from standard input when filtering (see
\sref{filtering}).

The \verb+-D+\oid{D}{define} option defines a variable.  Its argument must
be a valid definition statement per \sref{defnstmt}.  The definition may use
the \verb+=+, \verb+:=+ or \verb+?=+ operator.  Multiple \verb+-D+ options may
be given, and if so they are defined in the order they are given on the
command line.  If any \verb+-D+ option includes place notation, a \verb+-b+
option must be used to provide the number of bells.

The \verb+-m+\oid{m}{module} option tells \gsiril\ to import the module named
in option's argument, exactly as if it had been named in \ttcmdidx{import}
statement.  If multiple \verb+-m+ options are given, each module is imported
in the order given on the command line.

The \verb+--no-init-file+\loid{no-init-file} option disables reading of the
standard initialisation file listed in Appendix~\ref{initfile}.%
\footnote{Strictly speaking, the initialisation file is not read, as its
contents are hard-coded into the program.}  
This option is generally only useful when debugging, or to provide an
alternative initialisation file.

These options all affect the input which \gsiril\ processes.  When several of
these options are used concurrently, they are evaluated in the following
order:
\begin{enumerate}[itemsep=0pt,leftmargin=*,labelsep=1em,topsep=0pt]
\item Read the initialisation file (see Appendix~\ref{initfile}), unless the
\verb+--no-init-file+ option is used.
\item Define any symbols which were provided in \verb+-D+ options.
\item Import any modules which were specified in \verb+-m+ options.
\item Define \ttcmdidx{everyrow} to \verb+"@"+ if the \verb+-E+ option\oi{E}
is given.
\item\label{filterloop} If filtering, define the \ttcmdidx{m} and \ttcmdidx{lh}
symbols with the next method from the input stream per \sref{filtering}.
\item Process the source from standard input or as given in a \verb+-e+ or
\verb+-f+ option, unless the \verb+-N+ option is given. 
\item If filtering, return to step \ref{filterloop} until the input stream is
exhausted.
\end{enumerate}

\section{Configuration options}

\begin{tabularx}{\textwidth}{llX}
\texttt{-b}&\texttt{--bells=BELLS}&
  The default number of bells\\
\texttt{-n}&\texttt{--extents=NUM}&
  The number of extents required\\
\texttt{-r}&\texttt{--rounds=ROW}&
  The starting 'rounds'\\
&\texttt{--length=RANGE}&
  Require the touch length to be in the specified range\\
&\texttt{--row-mask=MASK}&
  Applies the specified row mask when printing a row
\end{tabularx}

The \verb+-b+, \verb+-n+, \verb+-r+, \verb+--length+ and \verb+--row-mask+
options\oid{n}{bells}\oid{n}{extents}\oid{r}{rounds}
serve exactly the same roles as the \ttcmdidx{bells},
\ttcmdidx{extents}, \ttcmdidx{rounds}, \ttcmdidx{rows} and \ttcmdidx{row\_mask}
statements, respectively, which are documented in \sref{configstmts}.

The argument to the \verb+--length+ option\loid{length} is either a number, or
a pair of numbers separated by a `\verb+-+' which serves was same role as the
word `\texttt{to}' in the \texttt{rows} statement.  For example, the
\verb+--length=1250-1450+ option has the same effect as the statement
`\texttt{1250 to 1450 rows}'.
% TODO: one end point of the range may be omitted

The arguments to the \verb+-r+\oid{r}{rounds} and \verb+--row-mask+
\loid{row-mask} options are respectively a row and a row pattern.  They are
not enclosed in quotation marks\footnote{Except for the purpose of quoting in
the shell.} or slashes, as the corresponding statements are.

None of options are required in most circumstances.  However, if either the
\verb+-r+ or \verb+--filter+ options\loi{filter} are used, then the \verb+-b+
option is required, and the row specified as the argument to \verb+-r+ must be
on the number of bells specified in the \verb+-b+ option. 

\section{Controlling output}
\label{outopts}

\begin{tabularx}{\textwidth}{llX}
\texttt{-i}&\texttt{--interactive}&
  Run in interactive mode\\
\texttt{-v}&\texttt{--verbose}&
  Run in verbose mode\\
\texttt{-q}&\texttt{--quiet}&
  Do not give truth output; or if given twice, no output at all\\
\texttt{-E}&\texttt{--everyrow-only}&
  Only print output from the everyrow symbol\\
\texttt{-R}&\texttt{--red=BELLS}&
  Colour \texttt{BELLS} in red\\
\texttt{-G}&\texttt{--green=BELLS}&
  Colour \texttt{BELLS} in green\\
\texttt{-B}&\texttt{--blue=BELLS}&
  Colour \texttt{BELLS} in blue\\
\end{tabularx}

The \verb+-i+ and \verb+-v+ options\oi{i}\oi{v} were documented in
\sref{interactive}.  They are normally used together.

\index{silent mode|(} 
The \verb+-q+ option\oid{q}{quiet} may be used multiple times to make 
\gsiril\ increasingly quiet.  If used once, \gsiril\ switches to silent mode
when executing the \ttcmdidx{true} symbol, meaning no statement of the truth is
given after a true touch.  If used twice, \gsiril\ switches to silent mode
when executing each \ttcmdidx{prove} and \ttcmdidx{echo} statement, disabling
all output during proof.

The \verb+-E+ option\oid{E}{everyrow-only} tells \gsiril\ to print every row
proved, while suppressing other output.  It does this by setting
\ttcmdidx{everyrow} to \verb+"@"+, and then switching to silent mode while
executing each \texttt{prove} statement (but not each \texttt{echo} statement:
add \verb+-qq+ for that).  Silent mode is temporarily disabled when the
\texttt{everyrow} symbol is automatically invoked.  The \ttcmdidx{row\_mask}
statement has no effect if the \verb+-E+ option has been given, though the
\verb+--row-mask+\loi{row-mask} option continues to work.\footnote{This
difference between the \texttt{row\_mask} statement and \verb+--row-mask+
option is because of the different ways in which they are used.  The statement
is typically used to suppress bells from lead or course heads.  However,
if the \verb+-E+ and \verb+--row-mask+ options are used together on the
command line, it is assumed this is intentional, and only rows matching the
supplied mask will be output, and with any fixed bells removed.}
\index{silent mode|)}
 
The \verb+-R+, \verb+-G+ and \verb+-B+\oid{R}{red}\oid{G}{green}\oid{B}{blue}
options are used to colour certain bells in red, green or blue, respectively,
when they are printed in rows using a \verb+@+ in a string.  The argument to
the option is one of more bell symbols, for example, \verb+-BET+ would colour
bells 11 and 12 blue.  If the argument contains a \verb+*+ then all subsequent
bells are additionally picked out in a bold or brighter colour.  Each \verb+*+
in the argument toggles the bold state on or off for any further bells.  Thus,
\verb+-R*1*2+ picks out both 1 and 2 in red, but with the treble bold.
These options are only supported if \gsiril\ was compiled with support for the
termcap or curses libraries, and if your terminal supports coloured output.%
\footnote{These facilities are common on Linux, but less so on Windows.}

\section{Compatibility modes}
\label{compatibility}

\begin{tabularx}{\textwidth}{llX}
\texttt{-I}&\texttt{--case-insensitive}&
  Treat symbols case-insensitively\\
\texttt{-P}&\texttt{--prove[=SYMBOL]}&
  Proves the specified symbol, or the first is none is specified\\
&\texttt{--msiril}&
  Run in MicroSiril compatibility mode\\
&\texttt{--sirilic}&
  Run in Sirilic compatibility mode\\
\end{tabularx}

The \verb+--msiril+ option\loid{msiril} actives \textitidx{MicroSiril
compatibility mode}.  MicroSiril compatibility mode is also activated
automatically if the program has been renamed \texttt{msiril} or
\texttt{microsiril}.\footnote{On Windows, an \verb+.exe+ extension is
allowed and the program name is treated case-insensitively.  The
alternatives \texttt{gmsiril} and \texttt{gmicrosiril} are also
supported.} 
Making a copy of \gsiril\ or symbolic link to it can be a convenient way
to active MicroSiril compatibility mode, and if you have compiled and
installed \gsiril\ from source, there should already be copy installed
as \texttt{msiril}.

MicroSiril compatibility changes the behaviour of \gsiril\ so that
compositions written for MicroSiril should behave identically when
proved with \gsiril.  It does not disable all functionality beyond what 
MicroSiril supported, if that functionality is invoked using syntax
that was not valid in MicroSiril.

In MicroSiril mode, comments begin with a single slash rather than a
pair of slashes,\index{comments}
 and symbol names may contain (but not begin with) a
`\verb+!+', `\verb+-+' or `\verb+%+'.  This conflicts with \gsiril's
syntax for row patterns, and without row patterns, the
\ttcmdidx{row\_mask}\ statement and alternatives blocks are of limited use:
these are all therefore disabled in MicroSiril mode.  Variable
interpolation in strings is also not supported, as this changes
the meaning of certain uses of a `\verb+$+' for recording the number of
duplicates.  MicroSiril mode also implies the \verb+-P+ option (without
an argument) and the \verb+-I+ option.

The \verb+-I+ option\oid{I}{case-insensitive} tells \gsiril\ to treat
symbols and keywords case insensitively.  Thus, for example, you cannot
use `\texttt{b}' to mean a bob and `\texttt{B}' to mean a course
containing a bob before.  This option is invoked implicitly in
MicroSiril compatibility mode.

The \verb+-P+ option\oid{P}{prove}, when used without an argument,
tells \gsiril\ to prove the first symbol defined directly in the input.%
\footnote{This does not include symbols defined in modules included
using \texttt{import} statement or the \verb+-m+ option, nor those defined in
the init file or using a \verb+-D+ during option.  The way this is implemented
is that after every definition, a symbol called \texttt{\_\_first\_\_}%
\index{first\_\_@{\hspace*{-\loptwidth}\texttt{\_\_first\_\_}}}
is conditionally set to the name of the symbol just defined, as if with
a statement `\texttt{\_\_first\_\_ ?= name}'.  The value of
\texttt{\_\_first\_\_} is saved when a module is imported and restored
afterwards.  This behaviour is an implementation detail which is subject
to change.}
It allows compositions to be proved without an explicit prove statement;
is was the behaviour of MicroSiril and is invoked in MicroSiril
compatibility mode.  When an argument is provided to the \verb+-P+
option, the argument is the name of a symbol to prove.  For example
\verb+-Ppeal+ is equivalent to adding the line `\texttt{prove peal}' to
the end of the source file. 

The \verb+--sirilic+\loid{sirilic} option invokes \textitidx{Sirilic
compatibility mode}, which aims to be compatible with the Sirilic
program written by Stephen Noyes\index{Noyes, Stephen} in 2002.  Siliric
compatibility mode is also invoked automatically if the program has been
renamed \texttt{sirilic} or \texttt{gsirilic}.  This has all the effects
of MicroSirlic mode, but also makes the underscore character toggle
underlining in a string, assuming the terminal supports it.  Even in
Sirilic compatibility mode, \gsiril\ does not currently support
Sirilic's extensions for analysing music or for checking which leads of
a method each bell rings, but this may be added in a future version.

\section{Filtering methods}
\label{filtering}

\begin{tabularx}{\textwidth}{lX}
\texttt{--filter}&
  Run as a filter on a method library or stream\\
\texttt{--lead-symbol=SYM}&
  Assign one lead's place-notation (by default excluding the lead head) to
  \texttt{SYM}; default symbol is `\texttt{m}'\\
\texttt{--lead-includes-lh}&
  Assign the whole lead's place notation, including the lead head, to
  the lead symbol\\
\texttt{--lh-symbol=SYM}&
  Assign the lead end change to \texttt{SYM}; default symbol is
  `\texttt{lh}'\\
\texttt{--payload-symbol=SYM}&
  Assign filter payload to \texttt{SYM}
\end{tabularx}

The \verb+--filter+ option\loid{filter} tells \gsiril\ to read methods
from standard input, runs the provided \gsiril\ code on each, switching to
silent mode\index{silent mode} for each \texttt{proof} statement.%
\footnote{The \texttt{echo} statement is not silenced while filtering:
use \verb+-qq+ for that.} and prints on standard output a list of those
methods where the \gsiril\ code executes successfully.  

Data is accepted on standard input as a \textitidx{method stream}, which
is a common format used to pass lists of methods between programs.  The
syntax is straightforward: methods are given, one per line, in place
notation.  Leading whitespace is permitted and ignored, but whitespace
is not permitted within the place notation.\footnote{\gsiril\ treats
space, tab, carriage return and line feed to be whitespace.  A UTF-8
byte order mark (BOM) is also permitted, though not required, at the
start of the first line.} 
% TODO: No computed bell expressions, e.g. {N-1}.  The '+' is optional.
% Commas allowed.
Any text after the place notation is called the \textitidx{filter payload},
and is copied through to \gsiril's output.  The \ttcmdidx{methsearch} program,
which is a tool for exhaustively searching for methods satisfying some
particular criteria, is often useful for providing a method stream for
\gsiril\ to filter.  The \ttcmdidx{methodlib} program, which is a tool for
listing methods in a method library, can also be useful.%
\footnote{Both \texttt{methsearch} and \texttt{methodlib} are distributed as
part of the Ringing Class Library.}

By default when filtering, \gsiril\ will remove the final change from
the provided place notation and assign this to a symbol called
\ttcmdidx{lh}, and assign the remainder of the place notation to a
symbol called \ttcmdidx{m}.  The names of the symbols use can be altered
with the \verb+--lh-symbol+\loid{lh-symbol} and \verb+--lead-symbol+%
\loid{lead-symbol} options, respectively.  The 
\verb+--lead-includes-lh+\loid{lead-includes-lh} option tells \gsiril\
not to remove the lead-head change and instead assigns the entire place
notation to the \texttt{m} symbol.  The filter payload is normally
ignored, but the \verb+--payload-symbol+\loid{payload-symbol} option
tells \gsiril\ to assign it to the specified symbol, after leading and
trailing whitespace has been trimmed.

Because \gsiril\ is reading methods from standard input when filtering,
it cannot also read the code to prove the touch from \gsiril.  Either
the \verb+-f+\oi{f} or \verb+-e+\oi{e} option is therefore required when
filtering.  A \verb+-b+\oi{b} option is also needed to tell \gsiril\ the
number of bells used in the place notation in the input stream.

The following is a simple example \gsiril\ being used as a filter.  The
\texttt{methsearch} command produces a list of all plain doubles methods with
a four-lead course, including asymmetric method which are false in the plain
course.  \gsiril\ filters this to find methods for which three fourth's place
bobs home is a true extent.\footnote{There are 30 such methods, which are all
palindromic second's place methods, which is a necessary and sufficient
requirement to ensure that three fourth's place bobs home is a true extent.
The use of the \gsiril\ filter could therefore be replaced with the 
\texttt{-s -m*2} options to \texttt{methsearch}.}

\begin{Verbatim}
methsearch -b5 | gsiril --filter -b5 -e 'prove 3(3(m,lh),m,+4)'
\end{Verbatim}

\section{Restricting functionality}

\begin{tabularx}{\textwidth}{lX}
\texttt{--disable-import}&
  Disable the \texttt{import} directive\\
\texttt{--node-limit=NUM}&
  Limit the prover to the specified number of nodes\\
\texttt{--prove-one}&
  Prove only one composition\\
\end{tabularx}

\gsiril\ has a number of options which restrict the functionality
available to a \gsiril\ composition.  These are useful for security
purposes\index{security} if the composition source comes from an
untrusted source, for example in an online prover.

The \verb+--disable-import+ option\loid{disable-import} disables the
import directive which allows a \gsiril\ script to read an arbitrary
file.

The \verb+--node-limit+ option\loid{node-limit} can be used to limit the
resources used by \gsiril.  During proof, the composition is parsed
into a tree structure which is executed recursively.  This option limits
the number of nodes in that tree which will be executed before the
program quits.  The example file describing Middleton's Cambridge in
\sref{example} involves about 15\,000 nodes and probably takes a few tens
of milliseconds to run on a modern computer.  \gsiril\ files which make
use of more complicated functionality may involve more nodes, but for
most purposes a limit of 1\,000\,000 will catch infinite loops while not
stopping genuine compositions from being proved.

The \verb+--prove-one+ option\loid{prove-one} tells \gsiril\ only to
execute one prove statement.  This is particularly useful with the
\verb+-E+ option if the output is being processed further, and the
program doing the processing wants to be sure it is only processing rows
from a single composition.

The online version of \gsiril\ on the BellBoard website runs with
\texttt{--prove-one --disable-import --node-limit=1000000}.
\index{BellBoard}


\section{Miscellaneous options}

\begin{tabularx}{\textwidth}{llX}
\texttt{-?}&\texttt{--help}&
  Print a help message and exit\\
\texttt{-V}&\texttt{--version}&
  Print version information and exit\\
\texttt{-L}&\texttt{--library=LIB}&
  Look up method names in the method library \texttt{LIB}\\
&\texttt{--determine-bells}&
  Determine the number of bells for each proof without actually proving them\\
&\texttt{--trace-all-symbols}&
  Print the row when any symbol is executed\\
&\texttt{--trace-symbol=SYM}&
  Print the row when \texttt{SYM} is executed\\
\end{tabularx}

The \verb+--help+\loi{help} option was mentioned in \sref{help}.  It prints a
summary of all the command line options which \gsiril\ knows about.

The \verb+--version+\oid{V}{version} option prints the version number of
\gsiril.  (In practice, this is less useful than it might be.  Currently,
formal releases of \gsiril\ are made very infrequently, and most copies
in use are development snapshots.)

The \verb+-L+\oid{L}{library} option provides a method library for \gsiril\ to
use when loading methods with the \ttcmdidx{load}, \ttcmdidx{loadm} and
\ttcmdidx{loadlh} functions.  The library is not actually loaded until the
first time one of these functions is called.

The \verb+--determine-bells+\loid{determine-bells} option makes \gsiril\ print
the number of bells on standard output whenever a \texttt{proof} statement
is run.  The output is in the form of a \texttt{bells} statement.  It implies
a \verb+-qq+ option, and suppresses an unsuccessful exit status if the touch
is false.

The \verb+--trace-all-symbols+\loid{trace-all-symbols} option makes \gsiril
print a line on standard output containing the row and symbol name separated
by a tab every time a symbol other than \texttt{everyrow} is executed outside
of speculative mode.  The row mask is ignored when printing rows in this
manner.
The \verb+--trace-symbol+\loid{trace-symbol} is similar, but only traces
the specified symbols.  If only one symbol is being traced, the symbol name is
not included on the trace output.
These options are useful for debugging complex \gsiril\ programs;%
\footnote{The use of tracing as a debugging tool is why neither trace option
implied \verb+-qq+.}
but they are also useful for getting a list of leads of each method when
proving spliced, which can be useful in determining whether a composition is 
all the work.\index{all the work}

\index{options|)}
\appendix
\chapter{Formal grammar}

This appendix contains the formal grammar for the \gsiril\ language,
expressed in Extended Backus-Naur Form%
\index{Backus-Naur form, extended}\footnote{This appendix specifically
uses the form of EBNF used in §6 of the XML 1.0 standard.}  
It is unlikely to be of interest to most readers.

\section{Statements}
\label{stmt_grammar}

\begin{Verbatim}[xleftmargin=0pt]
statement      ::=  defn_stmt | prove_stmt | bells_stmt | rows_stmt |
                    extents_stmt | rounds_stmt | row_mask_stmt | 
                    echo_stmt | import_stmt | version_stmt | exit_stmt

defn_stmt      ::=  name assign_op expression?
assign_op      ::=  "=" | "?=" | ":"

prove_stmt     ::=  "prove" expression
 
bells_stmt     ::=  integer "bells"
rows_stmt      ::=  integer ( "to" integer )? "rows"
extents_stmt   ::=  integer "extents"
rounds_stmt    ::=  "rounds" row
row_mask_stmt  ::=  row_mask" pattern

echo_stmt      ::=  "echo" (string | expression)?
import_stmt    ::=  "import" string
version_stmt   ::=  "version"
exit_stmt      ::=  "exit" | "quit" | "end"
\end{Verbatim}

\chapter{Initialisation file}\label{initfile}

\begin{Verbatim}[xleftmargin=0pt]
__summary__ = "# rows ending in @"

true         = __summary__, "Touch is true"
notround     = __summary__, "Is this OK?"
false        = __summary__, "Touch is false in $ rows"
tooshort     = __summary__, "Touch is too short: \",
                            "at least ${min_length} rows expected"
toolong      = "Touch exceeds maximum length of ${max_length} rows$$"
conflict     = __summary__, "Touch not completed due to false row$$"

rounds   =
start    =
finish   =
abort    =

// This gets overwritten by gsiril after reading the init file if the -E 
// option has been given.
everyrow = 
\end{Verbatim}

\clearpage
\phantomsection
\footnotesize
\printindex[default][\normalsize 
\addcontentsline{toc}{chapter}{Index}
Underlined page references refer to the main definition of that option
or variable.  Subjects are not generally indexed when they appear
in the discussion of a similarly-named option.  
]

\normalsize
\chapter*{Colophon}
\addcontentsline{toc}{chapter}{Colophon}

This manual was typeset using \XeTeX\index{Xetex@\XeTeX}, a document 
typesetting package written by Jonathan Kew.  It is derived from 
\LaTeX, written by Leslie Lamport, but includes support for the Unicode 
character set and OpenType fonts.
Both are derived from Donald Knuth's original \TeX\ typesetting engine.  
These are all open source products and are freely available on a variety of 
operating systems, include GNU/Linux. 

The main font used is Linux Libertine, an open source OpenType font designed
by Philipp Poll.  The sans serif font used in chapter and section headings
is Linux Biolinum, also designed by Philipp Poll.  Despite their names,
neither font is specific to Linux.  Equations are set using Donald Knuth's 
Computer Modern font.  The monospaced font used for code fragments and
row literals is the Computer Modern Teletype font, again by Donald Knuth.%
\index{fonts}

The ringing-related programs described in this document are available as
part of the Ringing Class Library project,\index{Ringing Class Library}
and \gsiril\ uses it for manipulation rows, changes and methods, 
reading method libraries, analysing music, proof and much more.  
The Ringing Class Library was started by Martin Bright\index{Bright, Martin} 
and contains contributions from Richard Smith and Mark Banner.

The source code for this manual is available from the Ringing Class Library
GitHub\index{GitHub} repository and is itself open source.  Any additions or
corrections to the manual will be gratefully received.\index{open source}

\vfil
\footnotesize
Permission is granted to copy, distribute and/or modify this
document under the terms of the GNU Free Documentation License,
Version 1.3 or any later version published by the Free Software
Foundation; with no Invariant Sections, no Front-Cover Texts and
no Back-Cover Texts.\index{licence}


\end{document}
